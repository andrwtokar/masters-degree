\begin{abstract}

    \begin{center}
        \large{Исследование методов доменной адаптации для улучшения распознавания ключевых точек на теле человека} \\
    \large\textit{Токарев Андрей Сергеевич} \\[1 cm]

    %Краткое описание задачи и основных результатов, мотивирующее прочитать весь текст

    \end{center}
    
    \begin{large}
    После достижения хороших результатов в решении задачи распознавания ключевых точек на теле человека и оценки его позы, возникла необходимость уменьшения затрат на улучшение результатов модели на целевых данных. Для решения этой проблемы исследователями были предложены некоторые методы доменной адаптации без учителя, которые позволили значительно ускорить процесс разработки готового решения для узконаправленных задач.
    \end{large}

	\begin{large}
	В рамках данной работы будет представлен анализ как различных моделей оценки позы по ключевым точкам, так и методы доменной адаптации моделей к целевым данным. Также будет представлен анализ работы одного из методов адаптации, хорошо показавшего себя в задачах детекции объектов и повторной идентификации человека.
    \end{large}
    


\end{abstract}
\newpage