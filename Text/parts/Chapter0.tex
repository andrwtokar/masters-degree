\section{Введение}
\label{sec:Chapter0} \index{Chapter0}

Здесь необходимо описать описание проблемы. Ее актуальность. Где это можно использовать и что делать. Заворожить читателя для прочтения твоей работы.
ОПИСАТЬ научную НОВИЗНУ работы.

Современные технологии машинного обучения и компьютерного зрения продолжают активно развиваться, находя применение в самых разнообразных областях. Одно из направлений, активно развивающихся в последние годы, является решение задачи распознавания ключевых точек на теле человека (Keypoint Detection) или оценка позы человека (Human Pose Estimation). На сегодняшний день решения данной задачи могут иметь множество практический применений.(, включая системы наблюдения, анимацию, медицинскую диагностику и интерактивные интерфейсы.)

Одной из возможностей использовать распознавание позы человека является виртуальная реальность. Оцифровка позы человека с помощью неросетей позволяет сэкономить на закупке дорогостоящий костюмов. Можно установить несколько камер, которые будут восстанавливать позу человека и переносить ее компьютерное пространство. При добавлении генеративных алгоритмов можно создавать всевозможные аватары и погрузиться в "Оазис" из фильма Стивена Спилберга "Первому игроку приготовиться".

Другим уже реальным применением данной технологии является рефери спортивных соревнований. Уже сейчас система полуавтоматического определения оффсайда помогает судьям футбольных матчей по всему миру. А работает она на распознавании ключевых точек часте тела, которыми футболист может сыграть в мяч и определяет были ли нарушены правила или гол был забит чисто.  (ССЫЛКА)

Недалеко отходя от спорта, можно применять оценку позы для создания личных тренеров прямо в телефоне. Проект MediaPipe (ССЫЛКА) предлагает уже сейчас возможности для использование его моделей на смартфонах для подсчета количества выполнений таких упражнений, как отжимания, приседания или подтягивания, а также дает возможность оценивать правильность различных асан в йоге.

ЕЩЕ ПРИМЕРОВ ПРО ИСПОЛЬЗОВАНИЕ ЗАДАЧИ ОЦЕНКИ ПОЗЫ

\hfill \break
Обучение модели и разработка алгоритма её работы представляют собой чрезвычайно сложный и трудоемкий процесс. Этот процесс требует значительных ресурсов, как со стороны специалистов, так и в плане вычислительной мощности. Сначала необходимо собрать и подготовить данные, затем обучить модель, настроить её параметры и протестировать на различных наборах данных, чтобы убедиться в её точности и надёжности. Это часто занимает много времени и требует значительных финансовых вложений. 

Данные, на которых модели обучаются имеют общий характер и могут не подходить под определенную узко-специализированную задачу. Например, в футболе не особо важны ключевые точки рук и лица, но важны ключевые точки тела, ног и общая точка головы. Для адаптации модели под данную задачу необходимо проделать гигантсикй объем работы по сбору данных, их фильтрации и разметке. Особенно выделяется последняя часть работы, так как она требует работы нескольких экспертов, которые расставят точки на каждой отобранной фотографии или кадре видео.

Для избежания дополнительных работ по адаптации широкоспециализированной модели к узкой задаче ученые задумались над вопросом приспосабливания модели к новым запросам. Таким образом появилась такая область науки о нейросетях, как адаптация модели к новым доменам данных (англ. Domain Adaptation). ЧТО ТАКОЕ ДОМЕНЫ. КАК МОЖНО АДАПТИРОВАТЬ. ЧТО МОЖНО ИЗ ЭТОГО ПОЛУЧАТЬ.

ДУМАЮ ТУТ ЕЩЁ СТОИТ СКАЗАТЬ ПРО ПЕРЕНОС ОБУЧЕНИЯ.

\hfill \break
В конце стоит описать план по главам и что там будет описано. Кратко и по делу.

В данной работе будет рассмотрено применение алгоритма доменной адаптации Progressive Unsupervised Learning для нескольких решений задачи распознавания ключевых точек на теле человека. Также будут представлены качественные и количественные результаты проведенного эксперимента.


\newpage