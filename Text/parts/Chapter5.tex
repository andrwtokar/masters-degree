\section{Эксперимент}
\label{sec:Chapter5} \index{Chapter5}

В данной главе будет описан эксперимент, как таковой.

\subsection{Описание эксперимента}

Постановка эксперимента. Что планируется сделать и какие результаты хочется получить. Какие метрики будем использовать и по каким метрикам будем сравнивать.

\subsubsection*{Выбор модели для эксперимента}

В рамках поставленного эксперимента поставлена задачи исследования работы алгоритма PUL для задачи распознавания ключевых точек для нескольких моделей:
\begin{enumerate}
\item HRNet
\item ViTPose
\item RTMPose ???
\item SimCC + ResNet ???
\end{enumerate}

Предложенные модели будут обучены на исходном датасете в течение 20 эпох. Таким образом будут получены базовые наборы весов, от которых и будет дпроводиться дальнейшее исследование.

\subsubsection*{Описание метода доменной адаптации}
Отбор точек будет произвед путем сравнения ЗНАЧЕНИЯ (ЗАМЕНИТЬ) с заранее заданным пороговым значением. ЗНАЧЕНИЕ будет отбираться двумя способами:
\begin{enumerate}
\item Средняя уверенность в предсказанных значениях\\
Для каждго предсказанного результата модель возвращает значение уверенности в своем предсказании. Усредняя это значении по всем ключевым точкам получаем среднюю уверенность для фотографии
\item Средняя уверенность для всех видимых точек\\
Так как увереннсть на точках, которые не видно на фотографии может сильно занижать среднее ЗНАЧЕНИЕ для всего результата, то принято решение отбрасывать эти значения при высчитываании ЗНАЧЕНИЯ
\end{enumerate}

По порогу уверенности будет отбираться набор значений с псевдоразметкой, на которой модель будет дообучаться. Таким образом будет проведено N (УКАЗАТЬ ТОЧНОЕ ЗНАЧЕНИЕ) итераций адаптации и сравнено значение результатов модели при различных способах отбора псевдоразметки. Также в рамках эксперимента произведено полное дообучение модели на размеченном целевом домене и результаты буду предоставлен для сравнения с адаптированными.

\subsubsection*{Метрики оценки качества распознавания}

ТАКЖЕ НЕОБХОИМО СКАЗАТЬ ОБ МЕТРИКАХ ОЦЕНКИ РАСПОЗНАВАНИЯ

\subsubsection*{Используемые ресурсы}

ТАКЖЕ НЕОБХОДИМО СКАЗАТЬ ОБ ИСПОЛЬЗУЕМЫХ РЕСУРСАХ


\subsection{Данные}

Для эксперимента требовалось найти два различных набора данных. В качестве исходного набора был отобран датасет COCO, который является основным для большинства современных моделей. Целевой набор был собран и размечен специально для данной задачи.

\subsubsection*{Исходные домен}

Здесь надо рассказать про датасет коко и привести немного сведений о нем.\\
Пример ФОТОграфий из датасета.\\
Рассказать про аннотацию тут.\\
Рассказать про отбор фотографий из-за граниченности ресурсов.

\subsubsection*{Целевой домен}

Описание собранных данных. Численный объем датасета. Возможно количество локаций и распределение по ним.

Необходимо предоставить данные по распределению данных между людьми, по количеству данных для обучения/тестирования. Предоставить данные по локациям. Предоставить картинки с примерами данных, которые были собраны.

Описание системы полуавтоматической разметки данных. Что там использовалось и как проходит разметка.

Для разметки собранных данных была создана система полуавтоматической разметки данных pose-markup (ССЫЛКА). Система состоит из двух частей: автоматическая разметка ключевых точек с помощью нейросети от MediaPipe и корректировка полученных данных экспертом.

На данный момент проект ориентирован на разметку позы только на видео файлах, так как это было необходимо реализовать в рамках эксперимента. В будущем планируется добавить возможность размечать точки и на одиночных изображениях.

Добавить коллаж примера рботы системы (3 фото: исходное, после автоматической разметки, после корректировки. Необходимо явно указать точки, которые были изменены.)

Численные данные о разметке данных. Сколько пришлось разметить, сколько пришлось скореектировать. Предоставить некоторые картинки по распределению корретировки данных.

В рамках разметки данных можель имела неточности, поэтому роль эксперта была существенна при валидации работы автоматической части системы. Процент фотографий на которых были проведены изменения: собрать статистику и предоставить результаты корректировки разметки: сколько суммарно точек было изменено, сколько точек имели изменения более 5\% от роста человека (метрика в бакалаврском дипломе есть), тоже самое для 15\%, три случая для среднего количества точек на фотографии, три случая для фотографий (может большинство изменений были на трети фотографий, а остальные не меняли), представить сводную таблицу изменений для точек по топографии КОКО, визуализировать сводную таблицу на картинке. 

\subsection{Результаты эксперимента}

Предоставить относительно сухо результаты эксперимента. Можно дать базовый анализ ситуации и того, что мы видим.

Необходимо предоставить результаты по времени обучения нейросетей, времени дообучения нейросетей. \\ 
Собрать данные по количеству ошибок до-после обучения. Собрать данные по количеству ошибок при изменении домена. \\ 
Данные по ресурсам, на которых обучались нейросетки.


\newpage