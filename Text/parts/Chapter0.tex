\section{Введение}
\label{sec:Chapter0} \index{Chapter0}

%%Здесь необходимо описать описание проблемы. Ее актуальность. Где это можно использовать и что делать. Заворожить читателя для прочтения твоей работы.
%%ОПИСАТЬ научную НОВИЗНУ работы.

Современные технологии машинного обучения и компьютерного зрения продолжают активно развиваться, находя применение в самых разнообразных областях. Одно из направлений, активно развивающихся в последние годы, является решение задачи распознавания ключевых точек на теле человека (Keypoint Detection) или оценка позы человека (Human Pose Estimation). В настоящее время методы решения этой задачи могут иметь разнообразные практические применения.

Одной из возможностей использовать распознавание позы человека является виртуальная реальность. Оцифровка позы человека с помощью нейросетей позволяет сэкономить на закупке дорогостоящий костюмов. Можно установить несколько камер, которые будут восстанавливать позу человека и переносить ее компьютерное пространство. При добавлении генеративных алгоритмов можно создавать всевозможные аватары и погрузиться в "Оазис" из фильма Стивена Спилберга "Первому игроку приготовиться".

Другим, уже вполне реальным, применением данной технологии является использование её в качестве рефери на спортивных соревнованиях. Уже сейчас система полуавтоматического определения офсайда активно помогает судьям футбольных матчей по всему миру. Эта система функционирует на основе распознавания ключевых точек частей тела футболистов, которыми они могут играть в мяч, что позволяет определить, были ли нарушены правила или гол был забит чисто \cite{fifaOffside}. Таким образом, технология существенно повышает точность и объективность судейства, уменьшая количество ошибок и спорных моментов в игре.

Продолжая тему спорта, следует отметить, что оценка позы может быть использована для анализа тренировок и создания персональных помощников. Уже существуют несколько решений, направленных на анализ вашей игры в большой теннис, которые способны оценивать текущие результаты, указывать на области, требующие улучшения, и предлагать рекомендации по коррекции техники \cite{IEEETennis, tennisTrener}. Существует также проект MediaPipe от Google, предоставляющий публичные интерфейсы для анализа спортивной активности на основе распознавания ключевых точек \cite{mediapipe}. Этот проект не ограничивается только оценкой и классификацией асан йоги. Он также включает функции для подсчёта количества повторяющихся упражнений, таких как отжимания, подтягивания и приседания. 

\hfill \break
Применений данной технологии можно придумать множество, но для их реализации необходима модель, которая будет работать быстро, поддерживая режим реального времени, а также демонстрировать высокие показатели точности своей работы. Однако обучение модели и разработка алгоритма её работы представляют собой чрезвычайно сложный и трудоемкий процесс. Этот процесс требует значительных ресурсов, как со стороны специалистов, так и в плане вычислительной мощности. Сначала необходимо собрать и подготовить данные, затем обучить модель, настроить её параметры и протестировать на различных наборах данных, чтобы убедиться в её точности и надёжности. Это часто занимает много времени и требует значительных финансовых вложений. 

В тоже время, датасеты, на которых обучаются модели часто имеют общий характер и могут вносить сильную погрешность в результаты при изменении общих характеристик входных данных. Например, модель, обученная распознавать объекты на дневных фотографиях, может показывать низкую точность на ночных снимках из-за разницы в освещении и визуальных характеристиках. Поэтому важно проводить дополнительное обучение модели на данных, соответствующих конкретной задаче. Этот процесс также требует значительных усилий, аналогичных созданию универсального решения. В связи с этим ученые задумались над тем, как можно уменьшить объем дополнительных работ по приспособлению модели к новым проблемам, не теряя в ее производительности. Таким образом появились алгоритмы доменной адаптации (англ. domain adaptation) и переноса обучения (англ. transfer learning).

Суть данных подходов состоит в том, чтобы преодолеть разрыв между исходным и целевым доменами данных. Это достигается путем выравнивания распределений данных, адаптации признаков и применения обученных моделей к новому домену с минимальной дополнительной обработкой. Таким образом, модели становятся более гибкими и способны эффективно работать в различных условиях, не требуя значительного объема новых данных или переработки архитектуры. В конечном итоге это не только ускоряет процесс внедрения, но и снижает затраты на разработку, поскольку уменьшает необходимость в полном цикле обучения новой модели.

\hfill \break
%%В конце стоит описать план по главам и что там будет описано. Кратко и по делу.

Стоит отметить, что возможность адаптировать нейронную сеть к целевому набору данных имеет особую важность в задаче распознавания ключевых точек на теле человека. Различные приложения требуют адаптации к специфическим условиям съемки, будь то освещение, ракурс или качество изображения. А в разных видах спорта, акцент может быть смещен на разные части тела: в футболе важны ноги и корпус, а в баскетболе — руки и верхняя часть туловища. Отсюда требуется возможность быстрого и дешевого улучшения качества работы нейронной сети. 

С учетом вышесказанного, напрашивается вывод, что тема данной работы является полезной и важной в нынешних реалиях. В \autoref{sec:Chapter4} произведен обзор как различных моделей распознавания ключевых точек на теле человека (см в \autoref{sec:Chapter4_PE}), так и некоторых методов доменной адаптации (см в \autoref{sec:Chapter4_DA}). Также проведен эксперимент по применению \textit{Progressive unsupervised learning (PUL)}, который хорошо себя показал в задачах детекции объектов и повторной идентификации, к оценке позы. Для него был собран и размечен целевой набор данных, описанный в \autoref{sec:Chapter5}. Результаты эксперимента дают ход дальнейшим исследованиям применения PUL, за счет вариативности способов отбора псевдо-разметки.

%ПОПРАВИТЬ ОКОНЧАНИЕ. КАК БУДТО НЕКРАСИВОЕ

\newpage