\section{Обзор существующих решений}
\label{sec:Chapter4} \index{Chapter4}

В данной главе необходимо провести анализ существующих моделей. Аргументировать выбор тех или иных моделей, методов. Подготовить теоретическую базу перед экспериментов.

\subsection{Обзор моделей для распознавания КТ}

\subsubsection{DeepPose}

Чисто для исторической справки. Если нужен будет объем.

\subsubsection{AlphaPose}

Чисто для исторической справки. Если нужен будет объем.

\subsubsection{OpenPose}

??? год. Проект от лаборатории ??? от института ???. 

Показала хорошие результаты при предыдущих сравнениях. Использует

\subsubsection{BlazePose}

Проект MediaPipe от гугл использует сеть архитектуры BazePose.

Показывает хорошие результаты. Имеет хорошие

\subsubsection{HRNet}

2019 год. Основная модель от проекта MMPose. Использует интересную архитектуру и хорошо показала себя при предыдущих сравнениях.

\subsubsection{ViTPose}

Трансформер. ??? год. Интересная архитектура. Рассказать про трансформер, рассказать про кодеки/докедеки. Рассказать про способ работы. Можно расписатьсся неплохо.

\subsubsection{RTMPose}

2023 год. На момент ислледования наиболее актуальная модель от проекта MMPose.

Работает неплохо. Обучается тоже неплохо. Попробуем заюзать в экспериментах.

\hfill \break

ОБЗОР ДАЛЬНЕЙШИХ РЕШЕНИЙ ПОЗЫ ПОД БОЛЬШИМ ВОПРОСОМ.

\subsubsection{Swin}

Трансформер. 2021 год. Все )

\subsubsection{Simcc + Resnet}

Вообще хз что это, но если смогу, то сделаю эксперимент )

\subsubsection{Dekr + HRNet}

Подход снизу вверх. ??? год. Модель, обученная лабораторией показывает хорошие результаты, но очень прожорлива в плане ресурсов для обучения.

\subsubsection{YoloPose}

Несколько различных версий архитектуры Yolo дают несколько различных версий модели для распознавания позы. Использует подход снизу-вверх и хорошо


\subsection{Анализ и выбор методов доменной адаптации}

Опираясь на модели из предыдущей главы надо сделать анализ и выбрать только несколько методов, которые применимы к нашим требованиям и которые мы будем сравнивать между собой.

\subsubsection{PUL}

Будем это использовать в работе. Надо будет хорошо расписать...

\subsubsection{RegDA}

Интересный алгоритм. На нем базируются все остальные

\hfill \break

ПРИ НЕХВАТКЕ ВРЕМЕНИ МОЖНО БУДЕТ ОСТАВИТЬ ТОЛЬКО ОДИН ИЗ СЛЕДУЮЩИХ МЕТОДОВ.

\subsubsection{UDA PoseEstimation}

Адаптация от синтетических данных к реальным. Должно бтыь интересно описать. Не использовали, так как до сих пор мы не перешли в 3х мерное пространство.

\subsubsection{POST}

Похожее на предыдущее

\subsubsection{SFDA}

Похожее на предыдущее


\newpage
