\section{Заключение}
\label{sec:Chapter6} \index{Chapter6}

%Необходимо рассказать о результатах исследования и о том, что мы получили. Привести сравнения, если таковые будут иметь место. Закинуть удочку для будущих исследований.

В рамках проведенного исследования были рассмотрены несколько решений задачи распознавания ключевых точек на теле человека, а также некоторые способы доменной адаптации без учителя. В качестве практической части работы была оценена применимость метода адаптации progressive unsupervised learning, основанного на генерации и фильтрации псевдо-размеченных данных, к моделям оценки позы. 

Для эксперимента был собран и частично аннотирован набор данных для оценки позы боксеров во время тренировки. Чтобы упростить автоматизацию процесса разметки была создана полуавтоматическая система pose-markup с использованием нейросетей, не задействованных в рамках исследования.

В результате имеются следующие выводы:

\begin{itemize}
\item При использовании в качестве функции фильтрации отсеивание поз по средней уверенности точек в ней, необходимо выбирать порог доверия не менее 0.5. Иначе в выборку будет попадать большой объем ошибочных данных, в которых модель не очень уверена, и результаты адаптации будут монотонно ухудшаться;
\item Наилучший результат адаптации был получен при использовании алгоритма SimCC  с магистральной нейронной сетью ResNet. В рамках доменной адаптации было повышено количество верно распознанных ключевых точек для метрики PCK с пороговым значением 0.5 и улучшена точность уже имеющихся предсказаний;
\item Отсутствие необходимости в разметке целевых данных экономит множество ресурсов, ведь генерация псевдо-разметок на домене объемом примерно 2200 изображений занимала не более 10 минут на итерацию, а аннотирование этих данных с помощью pose-markup заняло не менее 74 часов;
\item При наличии сложно различимых частей тела на изображениях стоит выбирать другую функцию фильтрации, так как предсказания этих ключевых точек будут сильно ухудшать возможности адаптации.
\end{itemize}

Дальнейшие планы развития данной темы включают в себя поиск новых методов фильтрации псевдо-разметок, основанных на анализе тепловых карт и их производных, сравнение результатов работы полученного метода с другими способами адаптации моделей к целевым доменам, расширение полученного набора данных новыми локациями и людьми.

\newpage