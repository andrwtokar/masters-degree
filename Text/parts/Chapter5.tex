\section{Эксперимент}
\label{sec:Chapter5} \index{Chapter5}

В данной главе будет описан эксперимент, как таковой.

\subsection{Описание деталей эксперимента}

Постановка эксперимента. Что планируется сделать и какие результаты хочется получить. Какие метрики будем использовать и по каким метрикам будем сравнивать.

\subsection{Данные. Сбор и разметка}

Описание собранных данных. Численный объем датасета. Возможно количество локаций и распределение по ним.

Необходимо предоставить данные по распределению данных между людьми, по количеству данных для обучения/тестирования. Предоставить данные по локациям. Предоставить картинки с примерами данных, которые были собраны.

Описание системы полуавтоматической разметки данных. Что там использовалось и как проходит разметка.

Для разметки собранных данных была создана система полуавтоматической разметки данных pose-markup (ССЫЛКА). Система состоит из двух частей: автоматическая разметка ключевых точек с помощью нейросети от MediaPipe и корректировка полученных данных экспертом.

На данный момент проект ориентирован на разметку позы только на видео файлах, так как это было необходимо реализовать в рамках эксперимента. В будущем планируется добавить возможность размечать точки и на одиночных изображениях.

Добавить коллаж примера рботы системы (3 фото: исходное, после автоматической разметки, после корректировки. Необходимо явно указать точки, которые были изменены.)

Численные данные о разметке данных. Сколько пришлось разметить, сколько пришлось скореектировать. Предоставить некоторые картинки по распределению корретировки данных.

В рамках разметки данных можель имела неточности, поэтому роль эксперта была существенна при валидации работы автоматической части системы. Процент фотографий на которых были проведены изменения: собрать статистику и предоставить результаты корректировки разметки: сколько суммарно точек было изменено, сколько точек имели изменения более 5\% от роста человека (метрика в бакалаврском дипломе есть), тоже самое для 15\%, три случая для среднего количества точек на фотографии, три случая для фотографий (может большинство изменений были на трети фотографий, а остальные не меняли), представить сводную таблицу изменений для точек по топографии КОКО, визуализировать сводную таблицу на картинке. 

\subsection{Результаты эксперимента}

Предоставить относительно сухо результаты эксперимента. Можно дать базовый анализ ситуации и того, что мы видим.

Необходимо предоставить результаты по времени обучения нейросетей, времени дообучения нейросетей. \\ 
Собрать данные по количеству ошибок до-после обучения. Собрать данные по количеству ошибок при изменении домена. \\ 
Данные по ресурсам, на которых обучались нейросетки.


\newpage